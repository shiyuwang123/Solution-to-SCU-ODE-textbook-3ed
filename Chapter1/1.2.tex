\documentclass[12pt]{article}
\usepackage{amsmath, amssymb}
\usepackage{enumitem}
\usepackage{geometry}
\geometry{a4paper, margin=1in}

\title{Solutions to 1.2 Exercises}
\author{Shiyu Wang(2022141500089)}
\date{\today}

\begin{document}

\maketitle

\section{exercise 1.2}

\begin{enumerate}[label=\textbf{Exercise \arabic*:}]
    \item
    \textbf{Problem:} \\
    Prove that for any constant $C$, the function $x(t)=Ce^{-3t}+2t+1$ is a solution of the ODE:
    $$\frac{dx}{dt}+3x=6t+5$$
    and find the solution satisfying the initial condition $x(0)=3$.
    
    \textbf{Solution:} \\
    First, we compute the derivative of $x(t)$:
    $$\frac{dx}{dt} = -3Ce^{-3t}+2.$$
    Now, we substitute $x(t)$ and $\frac{dx}{dt}$ into the left-hand side of the ODE:
    $$\frac{dx}{dt}+3x = (-3Ce^{-3t}+2)+3(Ce^{-3t}+2t+1).$$
    Simplifying this expression, we get:
    $$-3Ce^{-3t}+2+3Ce^{-3t}+6t+3 = 6t+5.$$
    Thus, the left-hand side equals the right-hand side of the ODE, confirming that $x(t)$ is indeed a solution.

    Next, we apply the initial condition $x(0)=3$:
    $$x(0) = Ce^{0}+2(0)+1 = C+1 = 3.$$
    Solving for $C$, we find $C=2$. Therefore, the specific solution satisfying the initial condition is:
    $$x(t) = 2e^{-3t}+2t+1.$$

    \item
    \textbf{Problem:} \\
    


    \textbf{Solution:} \\
    
    


    \vspace{1em}

    % Add more exercises as needed
\end{enumerate}

\end{document}